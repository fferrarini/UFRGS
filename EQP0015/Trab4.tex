\section{Objetivos}
O presente trabalho tem por objetivo construir um diagrama de equilíbrio
líquido-vapor (ELV) para uma mistura de um polímero (poliisobutileno - PIB) com
um solvente (benzeno). Para isso, será apresentada uma breve discussão dos
modelos e metodologia de cálculo, assim como a teoria aplicada.

\section{Introdução Teórica}


A teoria de \citeonline{Flory1951} e \citeonline{Huggins1942} considera a
energia livre de Gibbs da mistura do polímero puro com o solvente
($\Delta g_{mis}$) em termos de duas contribuições: entalpia de mistura 
($\Delta h_{mis}$) e entropia de mistura ($\Delta s_{mis}$), conforme é
apresentado a seguir:

\begin{equation}\label{eq:gemist}
\Delta g_{mis} = \Delta h_{mis} - T\Delta s_{mis}
\end{equation}
onde $T$ é a temperatura absoluta.

A entropia de mistura é determinada pelas frações volumétricas do solvente
($\phi_1$) e do polímero ($\phi_2$), enquanto a entalpia é determinada pelo
parâmetro adimensional de interação $\chi$, o qual normalmente é referido como parâmetro de interação de
Flory-Huggins, conforme é apresentado nas equações abaixo: 

\begin{equation}\label{eq:entalexc}
\Delta h_{mis} = \chi RT\left( x_1 + mx_2 \right)\phi_1\phi_2
\end{equation}

\begin{equation}\label{eq:entroexc}
\Delta s_{mis} = -R\left( x_1\ln\phi_1 + x_2\ln\phi_2 \right)
\end{equation}

Conforme é apresentado por \citeonline{Koretsky2013}, a energia de Gibbs em
excesso ($g^E$) é expressa através da diferença entre $\Delta g_{mis}$ e sua
condição ideal, como mostrado abaixo:

\begin{equation}\label{eq:geexc1}
g^E = \Delta g_{mis} - \Delta g_{mis}^{ideal}
\end{equation}
onde:

\begin{equation}
\Delta g_{mis}^{ideal} = RT\sum_ix_i\ln x_i
\end{equation}

Substituindo na \autoref{eq:geexc1}, obtem-se:

\begin{equation}
g^E = \Delta g_{mis} - RT\sum_ix_i\ln x_i
\end{equation}

Através da definição de $\Delta g_{mis}$ e considerando um sistema binário,
tem-se:

\begin{equation}\label{eq:geexc2}
g^E = \Delta h^{mist} - T\Delta s^{mist} - RT\left( x_1\ln x_1 + x_2\ln x_2
\right)
\end{equation}

Substituindo as Equações \ref{eq:entalexc} e \ref{eq:entroexc} na
\autoref{eq:geexc2} e fazendo-se as devidas simplificações, obtem-se uma
expressão para a energia de Gibbs em excesso em função das frações molares
($x_i$) e volumétricas ($\phi_i$), como é apresentado abaixo:

\begin{equation}\label{eq:geRT}
\frac{g^E}{RT} = \chi\left(x_1 + mx_2\right)\phi_1\phi_2 +
\left[x_1\ln\frac{\phi_1}{x_1} + x_2\ln\frac{\phi_2}{x_2}\right]
\end{equation}
onde $x_1$ e $x_2$ são as frações molares do solvente e do polímero,
respectivamente; $\phi_1$ e $\phi_2$ são, respectivamente, as frações
volumétricas do solvente e polímero, as quais são obtidas através das
seguintes equações:

\begin{equation}\label{eq:fracvap}
\phi_1 = \frac{n_1}{n_1 + mn_2}
\end{equation}

\begin{equation}
\phi_2 = \frac{mn_2}{n_1 + mn_2} = 1 - \phi_1
\end{equation}
onde $m$ é o número de repitições da unidade de monômero contida na cadeia
polimérica e pode ser obtida através da razão entre massa molar do polímero
($M_{wp}$) pela do monômero ($M_{wm}$), como é apresentado a seguir:

\begin{equation}
m = \frac{M_{wp}}{M_{wm}}
\end{equation}

 Aplicando-se a definição de uma
propriedade parcial molar para a \autoref{eq:geRT}, ou seja, realizando-se a
diferenciação com relação ao número de mols do solvente, obtem-se
a seguinte expressão para o coeficiente de atividade ($\gamma_1$) para tal
componente da mistura:


\begin{equation}\label{eq:gamma1}
\ln\gamma_1 = \ln\frac{\phi_1}{x_1} + \phi_2\left( 1 - \frac{1}{m} \right) +
\chi\phi_2^2
\end{equation}

De uma maneira análoga ao que foi feito para a obtenção da \autoref{eq:gamma1},
chega-se na seguinte expressão do coeficiente de atividade para o polímero
($\gamma_2$):

\begin{equation}
\ln\gamma_2 = \ln\frac{\phi_2}{x_2} - \phi_1\left( m - 1 \right) +
m\chi\phi_1^2
\end{equation}




Conforme é apresentado por \citeonline{Tadros2013}, o parâmetro de interação de
Flory-Huggins ($\chi$) fornece a medida da interação das cadeias poliméricas com as moléculas de solvente e, também, as interações
polímero-polímero. 

Quando $\chi$ assume valor igual a $1/2$, o polímero se comporta
como uma mistura ideal com o solvente e, nessa condição, é denotado por Flory
como ponto $\theta$. Desse modo, o polímero não sofre nenhum tipo de repulsão ou
atração. Para valore de $\chi$ inferiores a $1/2$, a mistura
se comporta como não ideal, apresentando desvios positivos (repulsão). 
Em contrapartida, para $\chi$ superior a $1/2$, ocorrem desvios negativos
(atração). Nessas condições, pode haver a precipitação do polímero.

Conforme é apresentado por \citeonline{Lindvig2002}, a relação entre o parâmetro
de interação de Flory-Huggins e o parâmetro de solubilidade é apresentada
através da seguinte expressão:

\begin{equation}
\chi = \frac{v_1}{RT}\left( \delta_1 - \delta_2 \right)^2
\end{equation}

Os parâmetros
de solubilidade dos componentes puros ($\delta_1$ e $\delta_2$), conforme
\citeonline{Kwak1986}, podem ser obtidos a partir da energia interna de
vaporização de cada componente puro $i$ ($\Delta u_i^v$), como um líquido saturado na temperatura do sistema
(\autoref{eq:deltai}):

\begin{equation}\label{eq:deltai}
\delta_i = \left ( \frac{\Delta u_i^V}{\nu_i^L} \right )^{\frac{1}{2}}
\end{equation}

O termo entre parânteses da \autoref{eq:deltai} é denominado de densidade de energia
coesiva e o parâmetro de solubilidade para misturas multicomponentes é mostrada
na \autoref{eq:delta}:

\begin{equation}\label{eq:delta}
\delta = \sum_i\sum_j\phi_i\phi_j\delta_{ij}^2
\end{equation}

onde

\begin{equation}\label{eq:deltaij}
\delta_{ij}^2 = \delta_i\delta_j
\end{equation}



\section{Metodologia}

Para a realização do estudo do equilíbrio líquido-vapor da mistura
polímero-solvetne contendo benzeno e poliisobutileno (PIB) utilizaram-se os
dados experimentais apresentados na \autoref{tab:dexp2}

\begin{table}[htb]
\renewcommand{\arraystretch}{1.3}
\centering
\caption{Dados de ELV da mistura polímero-solvente benzeno(1) / PIB(2) a 312,75
K.}
\begin{tabular}{S[table-format=2.2,round-mode=places,round-precision=2]
S[table-format=1.4,round-mode=places,round-precision=4]}
\toprule
{$w_1$ (\%)}	&	{P (bar)}	\\
\midrule

4,37	&	0,0715	\\
6,33	&	0,0971	\\
9,45	&	0,1236	\\
15,16	&	0,1681	\\
18,42	&	0,1818	\\
25,37	&	0,2095	\\
29,71	&	0,2182	\\
32,12	&	0,2207	\\
37,30	&	0,2267	\\
\bottomrule
\end{tabular}
\label{tab:dexp2}
\end{table}

As propriedades, necessárias para os cálculos, das duas substâncias
envolvidas no estudo estão mostradas na \autoref{tab:dexp1}.

\begin{table}[htb]
\renewcommand{\arraystretch}{1.3}
\caption{Propriedades dos componentes da mistura polímero-solvente benzeno(1) /
PIB(2)}
\sisetup{table-format=2.2,round-mode=places,round-precision=2}
\footnotesize
\center
\begin{tabular}{lr|lr}
\toprule
\multicolumn{2}{c}{Benzeno (1)}	&	\multicolumn{2}{c}{PIB (2)}		\\
\midrule 
{$M_{w1}\rm{ (g/mol)}$} 	&	{78}	&	{$M_{wp}\rm{ (g/mol)}$}	&	{40000}	\\
{$v_1\rm{ (cm^3/mol)}$}	&	{88,26}	&	{$M_{wm} \rm{ (g/mol)}$}	&	{104}	\\
{$P_1^{sat}\rm{ (bar)}$}	&	{0,2392}	&	{$v_m \rm{ (cm^3/mol)}$}	&	{131,9}	\\

\bottomrule
\multicolumn{4}{c}{Obs.: Dados retirados do material fornecido em aula}
\end{tabular}
\label{tab:dexp1}
\end{table}

Apenas para nível de comparação, a \autoref{tab:prop1} mostra alguns valores
experimentais de $\chi$ para a mistura benzeno/PIB para diferentes composições e
faixas de temperaturas.

\clearpage

\begin{table}[htb]
\centering
\renewcommand{\arraystretch}{1.3}
\caption{Valores experimentais de $\chi$ em função da temperatura e da fração
volumétrica do polímero ($\phi_2$)}
\begin{tabular}{lcc}
\toprule
{Temperatura ($^\circ$C)} & {Fração Volumétrica ($\phi_2$)} & {$\chi$}		\\
\midrule
{	10			}	&	{	0,4	a	0,8	}	&	{	0,670	a	0,920	}	\\
{	25			}	&	{	0,0	a	1,0	}	&	{	0,498	a	1,060	}	\\
{	25			}	&	{	1,0			}	&	{	0,880	a	0,610	}	\\
{	27	a	65	}	&	{	0,6	a	1,0	}	&	{	0,730	a	1,070	}	\\
{	30			}	&	{	0,0			}	&	{	0,495				}	\\
{	40			}	&	{	0,6	a	0,8	}	&	{	0,700	a	0,800	}	\\
{	50			}	&	{	0,0	a	0,2	}	&	{	0,485	a	0,583	}	\\
{	100			}	&	{	1,0			}	&	{	0,700				}	\\
{	37	a	200	}	&	{	1,0			}	&	{	1,180	a	0,700	}	\\

\bottomrule
\multicolumn{3}{c}{Fonte:\citeonline{Orwoll2007}}
\end{tabular}
\label{tab:prop1}
\end{table}

Para os cálculos de equilíbrio, utilizou-se o modelo da Lei de Raoult
Modificadas, considerando a fase vapor como sendo um gás ideal. O coeficiente de
atividade foi calculado através do mosdelo de Flrory-Huggins, explanado na seção
anterior.

\begin{equation}
\hat{f_1}^l = \hat{f_1}^v
\end{equation}

\begin{equation}
\gamma_1x_1P_1^{sat} = Py_1
\end{equation}

Como o solvente apresenta-se na forma pura na fase vapor, ou seja, sua fração
molar ($y_1$) é igual a 1,0, tem-se:

\begin{equation}
\gamma_1x_1P_1^{sat} = P
\end{equation}

Como os dados de composição foram apresentados em base mássica, foi necessário
convertê-los em base molar para fazer-se o uso do modelo. A seguir está mostrada
a relação entre as frações mássicas e molares:

\begin{equation}
x_1
=\frac{\displaystyle
\frac{w_1}{M_{w1}}}{\displaystyle\frac{w_1}{M_{w1}}+\frac{w_2}{M_{wp}}}
\end{equation}

A partir disso, é possível reescrever a \autoref{eq:fracvap} em função das
frações mássicas e volumes molares, como é apresentado a seguir:

\begin{equation}
\phi_1 =
\frac{\displaystyle\frac{w_1}{M_{w1}}v_1}{\displaystyle\frac{w_1}{M_{w1}}v_1+\displaystyle\frac{w_2}{M_{wp}}mv_m}
\end{equation}

Os códigos de programação foram implementados em
liguagem $Java$ utilizando o ambiente do Eclipse Mars 2 Release (4.5.2)

\clearpage

\section{Resultados}

\begin{table}[htb]
\centering
\renewcommand{\arraystretch}{1.3}
\caption{Valores dos erros médios (absoluto e relativo) para diferentes valores
do parâmetro de interação polímero-solvente ($\chi$)}
\begin{tabular}{ccc}
\toprule
& {Erro Médio} & {Erro Médio} 	\\
& {Absoluto} & {Relativo (\%)}	 \\
\midrule
{$\chi$} = 0,50 & 0,0431 & 29,06 \\
{$\chi$} = 0,80 & 0,0179 & 13,58 \\
{$\chi$} = 0,90 & 0,0105 & 8,61  \\
{$\chi$} = 0,95 & 0,0089 & 7,02  \\
{$\chi$} = 1,00 & 0,0092 & 6,38  \\
{$\chi$} = 1,05 & 0,0110 & 6,46  \\
{$\chi$} = 1,10 & 0,0147 & 7,81  \\
{$\chi$} = 1,20 & 0,0244 & 12,88 \\
{$\chi$} = 1,50 & 0,0643 & 38,34 \\
\bottomrule
\end{tabular}
\label{tab:allresult}
\end{table}

\section{Conclusões}