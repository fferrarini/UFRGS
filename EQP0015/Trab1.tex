\section{Introdução Teórica}

\subsection{Modelo UNIFAC}

\citeonline{Fredenslund1975} explicam que a metodologia UNIFAC 
(“Universal Quasi-chemical Functional Activity Coefficient”)
é utilizável para uma vasta quantidade de misturas exibindo 
desvios positivos ou negativos da lei de Raoult. O método 
UNIFAC segue o modelo “Analytical-Solution-of-Groups”, ou ASOG,
de Derr e Deal’s (1969), onde os coeficientes de atividade
em misturas são relacionados ás interações entre os 
grupos estruturais. Além disso, o método UNIFAC pode 
ser usado para prever os coeficientes de atividade para 
os componentes em misturas binárias desconhecidas que
fazem parte de um determinado sistemas multicomponentes. 
Tais previsões podem ser utilizadas na geração de parâmetros
 binários em qualquer modelo de excesso de Gibbs.

%\apudonline{Fredenslund1975}{Fredenslund1975}

\citeonline{Pollmann1996} citam que o logaritmo do coeficiente de
atividade em uma mistura é calculado (Equação 1) via o somatório de duas contribuições: a
combinatorial, devido aos diferentes tamanhos e formatos das moléculas, e a
residual representa a contribuição via interação energética entre as
moléculas.
\begin{equation}\label{eq:001}ln\gamma_i = ln\gamma_i^C +
ln\gamma_i^R\end{equation}

Onde
$ln\gamma_i^C$ é o termo combinatório e $ln\gamma_i^R$ 
é o termo residual.


Todavia, \apudonline{Gmehling1996}{Muzenda2013} cita que resultados não  
satisfatórios são obtidos para
sistemas cujos componentes detêm tamanhos bem diferentes, bem
como para coeficientes de atividade à diluição infinita. 
Assim, \citeonline{Muzenda2013} afirma que diferentes modificações 
foram propostas para os termos combinatorial e residual, 
bem como uma introdução da dependência da temperatura na 
interação dos parâmetros dos grupos funcionais.

\citeonline{Nagata1981} propuseram modificações apenas no termo residual
adicionando um termo tipo Flory-Huggins na relação de $ln\gamma_i^R$. 
Ademais, alterou a relação para com a fração de área 
superficial, utilizado originalmente, para dependência 
da fração molar do grupo funcional, além de relacionar o 
parâmetro de interação dos grupos funcionais com a 
temperatura. Entretanto, os resultados obtidos foram 
similares ao UNIFAC original, além de não haver grande 
evolução nos problemas encontrados no modelo original.

\citeonline{Larsen1987} propuseram modificações na expressão 
combinatorial do
UNIFAC original incorporando as modificações propostas por 
\citeonline{Kikic1980} e investigadas por \citeonline{Alessi1982}. 
O termo combinatorial tornou-se dependente do parâmetro de 
fração volumétrica do grupo funcional. O termo residual teve 
apenas modificação na interação com a temperatura. Como 
resultado, o modelo proposto obteve melhorias nas predições 
de equilíbrios líquido-vapor (VLE).

\citeonline{Weidlich1987} publicaram outro modelo modificado do
UNIFAC original, denominando-o de UNIFAC (Dortmund). Dentre as 
incorporações, destacam-se: a utilização dos parâmetros de 
volume e área superficial de van der Waals para alcanos 
cíclicos e reclassificação dos álcoois em primário, 
secundário e terciário com seus próprios parâmetros de 
volume e área superficial de van der Waals; e a extensão 
do ajuste dos parâmetros de interação dos grupos funcionais 
para inclusão dos coeficientes de atividade à diluição 
infinita, equilíbrios líquido-vapor e entalpia em excesso, 
buscando aprimorar a precisão de tais parâmetros.

Mais modelos baseados originalmente no UNIFAC de \citeonline{Fredenslund1975} são
propostos buscando, principalmente, melhorar a precisão dos 
parâmetros para mais tipos de equilíbrios e demais tipos de 
grupos funcionais. Neste trabalho, será utilizado o modelo 
UNIFAC (Dortmund) e, devido a tal, este modelo é melhor 
explicado na seção a seguir.

\subsection{Modelo UNIFAC Dortmund (Do)}

Segundo \citeonline{Jakob2006}, a contribuição do método 
UNIFAC (Dortmund), ou
UNIFAC (Do), é um modelo da energia livre de Gibbs em 
excesso ($g^E$), que permite a predição dos coeficientes de atividade em
sistemas não-eletrolisados, em função da temperatura e 
composição. O coeficiente de atividade é calculado via 
soma da parcela combinatorial e residual, para cada 
componente $i$, via Equação 1.

\citeonline{Jakob2006} cita que a parte combinatorial 
independente da temperatura é
calculada com os valores do volume via equação de van 
de Waals ($R_k$) e da área superficial ($Q_k$) dos grupos funcionais. 
Comparativamente com o modelo UNIFAC original, o UNIFAC (Do)
 apresenta uma mudança empírica na parte combinatorial ($V'$) 
para melhor descrever os sistemas assimétricos, como 
mostrado nas equações a seguir:

\begin{equation}\label{eq:002}
ln\gamma_i^C = 1 - V'_i + ln(V'_i) - 5q_i\left [ 1
- \frac{V_i}{F_i} + ln\left ( \frac{V_i}{F_i} \right ) \right ]
\end{equation}

\begin{equation}\label{eq:003}
V'_i = \frac{r_i^{3/4}}{\displaystyle\sum_{j=1}^nx_jr_j^{3/4}}
\end{equation}

\begin{equation}\label{eq:004}
V_i = \frac{r_i}{\displaystyle\sum_{j=1}^nx_jr_j}
\end{equation}

\begin{equation}\label{eq:005}
F_i = \frac{q_i}{\displaystyle\sum_{j=1}^nx_jr_j}
\end{equation}

O volume e área superficial relativos de van der Waals, para cada molécula $i$,
podem ser calculados pelas propriedades $R_k$  e $Q_k$  dos grupos estruturais
$k$ :

\begin{equation}\label{eq:006}
r_i = \displaystyle\sum_{k=1}^nv_k^{(i)}R_k
\end{equation}

\begin{equation}\label{eq:007}
q_i = \displaystyle\sum_{k=1}^nv_k^{(i)}Q_k
\end{equation}

A parte residual pode ser obtida utilizando os coeficientes de atividade dos
grupos funcionais $k$  na mistura ( $\Gamma_k$ ) e dos mesmos quando em uma
solução referência contendo apenas moléculas do tipo $i$ ( $\Gamma_k^{(i)}$ ):


\begin{equation}\label{eq:008}
ln\gamma_i^R = \displaystyle\sum_{k=1}^nv^{(i)}_k\left ( ln\Gamma_k -
ln\Gamma_k^{(i)} \right )
\end{equation}

A dependência da concentração do coeficiente de atividade do grupo funcional
$\Gamma_k$ pode ser descrito via \autoref{eq:009}:

\begin{equation}\label{eq:009}
ln\Gamma_k = Q_k\left ( 1 - ln\left ( \displaystyle\sum_m\Theta_m\Psi_{mk}
\right ) -
\displaystyle\sum_m\frac{\Theta_m\Psi_{km}}{\displaystyle\sum_n\Theta_n\Psi_{nm}}
\right )
\end{equation}
onde a fração de superfície ($\Theta_m$) e a fração molar ($X_m$) são definidas
nas Equações \ref{eq:010} e \ref{eq:011}.

\begin{equation}\label{eq:010}
\Theta_m = \frac{Q_mX_m}{\displaystyle\sum_nQ_nX_n}
\end{equation}

\begin{equation}\label{eq:011}
X_m =
\frac{\displaystyle\sum_jv^{(j)}_mx_j}{\displaystyle\sum_j\sum_nv_n^{(j)}x_j}
\end{equation}

A dependência da temperatura do parâmetro de interação entre os grupos
funcionais é descrita na \autoref{eq:012}, onde a interação é descrita entre os
grupos $n$ e $m$ .

\begin{equation}\label{eq:012}
\Psi_{nm} = exp \left ( \frac{-a_{nm} + b_{nm}T + c_{nm}T^2}{T} \right)
\end{equation}



