% Arquivo de tema para apresenta��o.
% N�o modifique este arquivo.

\usepackage[latin1]{inputenc}
\usepackage[english]{babel}
\usepackage{hyperref}
\usepackage{tikz}
\usepackage[]{qrcode}
% work with pdflatex
\tikzset{
  every overlay node/.style={anchor=south east,},
}
% Usage:
% \tikzoverlay at (-1cm,-5cm) {content};
% or
% \tikzoverlay[text width=5cm] at (-1cm,-5cm) {content};
\def\tikzoverlay{%
   \tikz[baseline,overlay]\node[every overlay node]
}%

\newcommand{\qrcodelabel}[3]{
   \begin{minipage}{#1in} \centering
   \qrcode[hyperlink,height=#1in]{#2}
   \tiny #3
   \end{minipage}
}

\newcommand{\qrcodelink}[2]{
   \begin{minipage}{#1in} \centering
   \qrcode[hyperlink,height=#1in]{#2}
   \tiny \href{http://#2}{#2}
   \end{minipage}
}

\newcommand{\qrcodeonly}[2]{
  \qrcode[hyperlink,height=#1in]{#2}
}

% symbols on footnotes and not numbers
\long\def\symbolfootnote[#1]#2{\begingroup%
\def\thefootnote{\fnsymbol{footnote}}\footnote[#1]{#2}\endgroup}
%\usepackage[side, symbol*]{footmisc}

\mode<handout>{
\usepackage{pnup}
\pgfpagesuselayout{3 on 1 with notes}[a4paper, border shrink=4mm]
\usepackage{beamerthemeBoadilla}
}

\mode<beamer>{
\usepackage{beamerthemeWarsaw}
\setbeamertemplate{blocks}[rounded][shadow=true]
\beamertemplatetransparentcovereddynamic
\beamertemplateballitem
\beamertemplatenumberedballsectiontoc
\hypersetup{
  %pdftitle={\title},
  %pdfauthor={\author},
  %pdfpagemode=FullScreen
  }
  %\pdfcompresslevel9
}

\usepackage{subfigure}

\mode<article>{
\usepackage{fullpage}
\usepackage{beamerarticle}
\usepackage{pgf}
\textwidth 0.6\textwidth
}

\pgfdeclareimage[width=6cm]{logobig}{LogoLVPP22}
\pgfdeclareimage[width=1cm]{logo}{LogoLVPP22}
\logo{\pgfuseimage{logo}}

\newenvironment{vrpresentation} {%start begin def
\mode<beamer>{
% Start page
% \frame[plain]{ %
% \frametitle{Instru��es para Visualizar esta Apresenta��o} %
% \begin{itemize} %
%   \item Esta apresenta��o � melhor visualizada no %
%      \href{http://www.adobe.com/products/acrobat/readstep2.html} %
%          {\beamergotobutton{Acrobat Reader\textregistered}} %
%    \item Preferencialmente utilize o modo %
%       \beamergotobutton{\Acrobatmenu{FullScreen}{tela cheia}} %
%    \item Avance na apresenta��o com a tecla \beamergotobutton{Page Down}, %
%       retorne com a tecla \beamergotobutton{Page Up} %
%     \item Durante a apresenta��o, navegue clicando nos item da barra de navega��o %
%       na parte superior ou inferior. %
% \end{itemize} %
% \vspace{1cm} %
% \begin{center} %
% \pgfuseimage{logobig} %
% \end{center} %
% }
} % \mode<beamer>
% P�gina de t�tulo
\mode<presentation>{
\frame[plain]{ %
\label{frame:title} %
\transdissolve %
\titlepage} %
% Sumario inicial
% \frame{ %
% \transdissolve %
% \frametitle{Summary} %
% \tableofcontents %
% } %
} %
\mode<article>{\maketitle}
%end begindef
} %
{ %begin enddef
\mode<beamer>{
% P�gina final.
% \frame[plain]{ %
% 	\frametitle{Obrigado!} %
% 	\begin{itemize} %
% %		\item O que voc� deseja fazer agora? %
% %			\begin{itemize} %
% 				\item \beamergotobutton{\href{http://www.enq.ufrgs.br/labs/lvpp}{visitar www.enq.ufrgs.br/labs/lvpp}} %
% 				\item \beamergotobutton{\hyperlink{frame:title}{Reiniciar a apresenta��o}} %
% 				\item \beamergotobutton{\Acrobatmenu{Quit}{Fechar esta apresenta��o}} %
% %			\end{itemize} %
% 	\end{itemize} %
% } %
%end enddef
}
\mode<article>{\begin{center} \pgfuseimage{logobig} \end{center} }
}

\usepackage{color}
\usepackage{listings}
\newcommand{\code}[1]{\texttt{#1}}

% pacote padrao ABNT para bibliografia
%\newlength{\bibindent}
%\usepackage[alf, abnt-etal-cite=2, abnt-etal-list=0]{abntcite}
%\bibliographystyle{abnt-alf}

% URL-like things:
\usepackage{url}
\newcommand{\email}{\begingroup \urlstyle{rm}\Url}
\newcommand{\rpsURL}{\url{www.rps.eng.br}}

% Chemical formula
\newfont{\elvsy}{cmsy10 scaled 1095} % 11pt
\newcommand{\chemical}[1]{{\begin{eqnarray}\fontdimen16\elvsy=3.3pt
\fontdimen17\elvsy=3.3pt\mathrm{#1}\end{eqnarray}}}
\newcommand{\chemicaltab}[1]{{$\fontdimen16\elvsy=3.3pt\fontdimen17\elvsy=3.3pt\mathrm{#1}$}}
\newcommand{\chemicaleqn}[1]{{\fontdimen16\elvsy=3.3pt\fontdimen17\elvsy=3.3pt\mathrm{#1}}}
