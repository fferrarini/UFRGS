
\documentclass[3p]{elsarticle}
\journal{Fluid Phase Equilibria}
\bibliographystyle{elsarticle-num-names}
\usepackage[portuguese]{nomencl}
% Hifenação e caracteres especiais
% português-brasil e inglês
% ---
\usepackage[brazil]{babel}
\addto\captionsbrazil{%
%  \renewcommand{\listfigurename}{Lista de Ilustra\c{c}\~{o}es}
}
\usepackage[utf8]{inputenc}
\usepackage[T1]{fontenc}

\usepackage{listings}	% Pacote para linguagens
\let\oldlstlistoflistings\lstlistoflistings
\renewcommand{\lstlistoflistings}{%
  \begingroup%
  \let\oldnumberline\numberline%
  \renewcommand{\numberline}{\lstlistingname~\oldnumberline}%
  \oldlstlistoflistings%
  \endgroup}
  
\usepackage[brazil]{hyperref}
 
% custom packages
%%
%%   Preambulo: pacotes, configura��es e comandos espec�ficos do usu�rio
%%
%% Rafael de Pelegrini Soares (rafael@rps.eng.br)
%% - Fev 2006

\usepackage{hyperref}
\def\sectionautorefname{Section}
%PDF LaTeX
\newif\ifpdf
\ifx\pdfoutput\undefined
  \pdffalse
\else
  \pdfoutput=1
  \pdftrue
\fi

\ifpdf
  %\hypersetup{colorlinks,citecolor=red,urlcolor=blue,pdfstartview=FitH}
\fi

\newcommand{\eqref}[1]{(\ref{#1})}

\usepackage{color}
\usepackage{listings}
\usepackage{subfigure}
\usepackage{graphicx}
\usepackage{pifont}
% \usepackage{geometry}
\usepackage{fleqn}
\usepackage{txfonts}





%\usepackage[style=authortitle-icomp,natbib=true,sortcites=true,block=space]{biblatex}

% float to the end of document (draft for revision)
%\usepackage[tablesfirst]{endfloat}

\usepackage{natbib}
\usepackage{soul}


\begin{document}
\begin{frontmatter} 
\title{An alternative reference fluid for the COSMO-SAC model}
\author{Rafael de P. Soares\corref{cor1}}
\ead{rafael.pelegrini@ufrgs.br}

\address{Departamento de Engenharia Qu\'imica, Escola de
Engenharia, Universidade Federal do Rio Grande do Sul,
Rua Engenheiro Luis Englert, s/n, Bairro Farroupilha, CEP 90040-040, Porto
Alegre, RS, Brazil\\July 2013}

\cortext[cor1]{Corresponding author. Tel.:+55 51 33083528; fax: +55 51 33083277}

\end{frontmatter}
% \section*{Introduction} \label{sec:intro}

A partir da equação deduzida para a energia de Helmholtz \emph{(A)},  é possível
expressar essa grandeza em termos da função de partição canônica, \emph{Q(N)}.
Sabe-se que essa grandeza também pode ser obtida através da soma dos potenciais
químicos de cada elemento do sistema, conforme é apresentado a seguir:
\begin{equation}\label{eq:hn}
A(N) = -kT\ln Q(N) = \mu_1 + \mu_2 + \ldots+ \mu_{m-1} + \mu_m + \ldots
+ \mu_{n-1} + \mu_{n} + \ldots + \mu_N
\end{equation}

De uma maneira análoga, para um sistema contendo \emph{N-2} elementos, a energia
de Helmholtz é calculada da seguinte maneira:
\begin{equation}\label{eq:hn-2}
A(N-2) = -kT\ln Q(N-2) = \mu_1 + \mu_2 + \ldots+ \mu_{m-1} + \ldots
+ \mu_{n-1} + \ldots + \mu_N
\end{equation}
Se for subtraída a \autoref{eq:hn-2} da \autoref{eq:hn}, obtém-se, então, uma
expressão que relaciona a razão entre as funções de partição desses dois
sistema com os potenciais químicos dos elementeos \emph{m} e \emph{n}, conforme
segue:
% \begin{equation}
% A(N) - A(N-2) = -kT\ln Q(N) + kT\ln Q(N-2) = \mu_m + \mu_n
% \end{equation}
\begin{equation}
\frac{Q(N-2)}{Q(N)} = \exp\left(\frac{\mu_m + \mu_n}{kT}\right)
\end{equation}

A probabilidade de se encontrar o par \emph{(m,n)}
\begin{equation}
p(m,n) = \frac{N_{mn}}{N_p} =
\displaystyle\frac{\exp\left(-\displaystyle\frac{E_{mn}}{kT}\right)Q(N-2)}{Q(N)}
\end{equation}

\begin{equation}
p(m,n) = \exp\left(\displaystyle\frac{-E_{mn}+\mu_m+\mu_n}{kT}\right)
\end{equation}

\begin{equation}
p(m) = \displaystyle\sum_np(m,n) =
\displaystyle\sum_n\exp\left(\displaystyle\frac{-E_{mn}+\mu_m+\mu_n}{kT}\right)
\end{equation}

\begin{equation}
\frac{\mu_m}{kT} = \ln p(m) -
\ln\displaystyle\sum_n\exp\left(\displaystyle\frac{-E_{mn}+\mu_n}{kT}\right)
\end{equation}

\begin{equation}
\frac{\mu_m^{\circ}}{kT} = \ln p(m) -
\ln\displaystyle\sum_n\exp\left(\displaystyle\frac{\mu_n^{\circ}}{kT}\right)
\end{equation}

\begin{equation}
\frac{\mu_m^{\circ}}{kT} = \ln p(m)
\end{equation}

\begin{equation}
\frac{\mu_n^{\circ}}{kT} = \ln p(n)
\end{equation}

\begin{equation}
\displaystyle\sum_np(n)=1
\end{equation}

\begin{equation}
\ln\Gamma_m = \frac{\mu_m - \mu_m^{\circ}}{kT}
\end{equation}

\begin{equation}
\ln\Gamma_m = -
\ln\displaystyle\sum_n\exp\left(\displaystyle\frac{-E_{mn}+\mu_n}{kT}\right)
\end{equation}

\begin{equation}
\frac{\mu_n}{kT} = \ln\Gamma_n + \frac{\mu_n^{\circ}}{kT}
\end{equation}

\begin{equation}
\ln\Gamma_m = -
\ln\displaystyle\sum_n\exp\left(\displaystyle\frac{-E_{mn}+\mu_n^{\circ}+\ln\Gamma_n}{kT}\right)
\end{equation}

\begin{equation}
\ln\Gamma_m = -
\ln\displaystyle\sum_np(n)\Gamma_n\exp\left(\displaystyle\frac{-E_{mn}}{kT}\right)
\end{equation}

\bibliography{ppgeq}

\end{document}
