
\documentclass[3p]{elsarticle}
\journal{Fluid Phase Equilibria}
\bibliographystyle{elsarticle-num-names}

% custom packages
%%
%%   Preambulo: pacotes, configura��es e comandos espec�ficos do usu�rio
%%
%% Rafael de Pelegrini Soares (rafael@rps.eng.br)
%% - Fev 2006

\usepackage{hyperref}
\def\sectionautorefname{Section}
%PDF LaTeX
\newif\ifpdf
\ifx\pdfoutput\undefined
  \pdffalse
\else
  \pdfoutput=1
  \pdftrue
\fi

\ifpdf
  %\hypersetup{colorlinks,citecolor=red,urlcolor=blue,pdfstartview=FitH}
\fi

\newcommand{\eqref}[1]{(\ref{#1})}

\usepackage{color}
\usepackage{listings}
\usepackage{subfigure}
\usepackage{graphicx}
\usepackage{pifont}
% \usepackage{geometry}
\usepackage{fleqn}
\usepackage{txfonts}





%\usepackage[style=authortitle-icomp,natbib=true,sortcites=true,block=space]{biblatex}

% float to the end of document (draft for revision)
%\usepackage[tablesfirst]{endfloat}

\usepackage{natbib}
\usepackage{soul}


\begin{document}
\begin{frontmatter}
\title{An alternative reference fluid for the COSMO-SAC model}
\author{Rafael de P. Soares\corref{cor1}}
\ead{rafael.pelegrini@ufrgs.br}

\address{Departamento de Engenharia Qu\'imica, Escola de
Engenharia, Universidade Federal do Rio Grande do Sul,
Rua Engenheiro Luis Englert, s/n, Bairro Farroupilha, CEP 90040-040, Porto
Alegre, RS, Brazil\\July 2013}

\cortext[cor1]{Corresponding author. Tel.:+55 51 33083528; fax: +55 51 33083277}

\end{frontmatter}
%\section*{Introduction}
%\label{sec:intro}

In the COSMO-SAC \citep{Lin:2002} model of \citeauthor{Lin:2002}, the \emph{activity coefficient} of a
segment $\Gamma_m$ is obtained using a reference fluid where all segments
have an identical chemical potential $\mu^0(0)$.
In the present work we show a new derivation for the activity coefficient of a segment
using the reference of an \emph{ideal} fluid with zero interaction energies but
not identical chemical potentials.

We start from the Eq.~(9) of \cite{Lin:2002}, which is also similar to the one originally derived
by \citeauthor{Klamt:1995}\cite{Klamt:1995}:
\begin{equation}\label{eq:mu}
	\frac{\mu_m}{kT} =-\ln\left[\sum_n
	\exp\left(\frac{-E_{m,n} + \mu_n}{kT}\right)\right] + \ln{p_m}
\end{equation}
where $\mu_m$ is the chemical potential of a segument $m$; $E_{m,n} = E_{n,m}$ is the interaction
energy for the contact between the segments $m$ and $n$; $p_m$ is the
probabylity of finding a segment $m$ in the mixture; and $T$ is the temperature.

Now let us consider an \emph{ideal} fluid where all the interations are zero, $E_{m,n}=0$,
resulting in:
\begin{equation}\label{eq:mu0}
	\frac{\mu_m^0}{kT} =-\ln\left[\sum_n
	\exp\left(\frac{\mu_n^0}{kT}\right)\right] + \ln{p_m}
\end{equation}

One could argue that, if there is no interaction, the chemical potential
of a segment $\mu_m^0$ should not depend on the chemical potential
of other segments $\mu_n^0$.
Indeed, since $\sum_n p_n = 1$, one can verify that $\mu_m^0/kT =
\ln{p_m}$ is a solution of Eq.~(\ref{eq:mu0}).

Then, if we subtract $\mu_m^0/kT = \ln{p_m}$ from Eq.~(\ref{eq:mu})
and define the activity coefficient of a segment as
$\ln\Gamma_m \equiv (\mu_m - \mu_m^0)/kT$, we get:
\begin{equation}\label{eq:lnGAMMA}
	\ln\Gamma_m =-\ln\left[\sum_n
	\exp\left(\frac{-E_{m,n} + \mu_n}{kT}\right)\right]
\end{equation}

Finally, if we use $\mu_n/kT = \ln\Gamma_n + \mu_n^0/kT =
\ln\Gamma_n + \ln p_n$, it is possible to show that:
\begin{equation}\label{eq:lnGAMMA}
	\ln\Gamma_m =-\ln\left[\sum_n p_n \Gamma_n
	\exp\left(\frac{-E_{m,n}}{kT}\right)\right]
\end{equation}
which is equivalent to Eq.~(10) of \cite{Lin:2002}.

With this derivation it is more clear that many contributions can be added directly
to the interaction energy $E_{m,n}$, not only the usual electrostatic and
hydrogen bond contributions.
Further, that the activity coefficient of a segment $\ln\Gamma_m$ can be used
to compute the residual Helmholtz energy, with respect to a non-interacting
fluid.

\bibliography{ppgeq}

\end{document}
