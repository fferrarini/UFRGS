
\documentclass[3p]{elsarticle}
\journal{Fluid Phase Equilibria}
\bibliographystyle{elsarticle-num-names}

% custom packages
\documentclass[a4paper,11pt]{article}

\usepackage[T1]{fontenc}
\usepackage{fancyhdr}
\pagestyle{fancy}
\usepackage{ae}
\usepackage[utf8]{inputenc}
% \usepackage[latin1]{inputenc}
\usepackage[brazil]{babel} \usepackage[brazil]{hyperref}
\usepackage[alf]{abntex2cite}
\usepackage{listings}
\usepackage{lscape}
\usepackage{longtable}
 
\usepackage{graphicx}
\usepackage{url}
\usepackage{lastpage}
\usepackage{multirow}
\usepackage{indentfirst}
\usepackage{amsmath}
\usepackage{siunitx}
\usepackage{booktabs}
\usepackage{pgfplots}
\usepackage{tikz}
\usepgfplotslibrary{colormaps} 
\usepackage{xifthen}
%\usepackage[miktex]{gnuplottex}
%\usepackage{gnuplottex}

\usepackage{subfloat}
\usepackage{float} 
\usepackage{subfig}
\usepackage{varwidth}
\newcommand{\subfigref}[1]{\hyperref[#1]{Figura~\ref*{#1}}}

\sisetup{output-decimal-marker = {,}}
\newcommand{\theauthorii}{}
\newcommand{\theauthoriii}{}
\newcommand{\theauthoriv}{}
\renewcommand{\title}[1]{\newcommand{\thetitle}{#1}}
\renewcommand{\author}[1]{\newcommand{\theauthor}{#1}}
\newcommand{\authorii}[1]{\renewcommand{\theauthorii}{#1}}
\newcommand{\authoriii}[1]{\renewcommand{\theauthoriii}{#1}}
\newcommand{\authoriv}[1]{\renewcommand{\theauthoriv}{#1}}

\newcommand{\class}[1]{\newcommand{\theclass}{#1}}
\newcommand{\theclassProffessorii}{}
\newcommand{\theclassProffessoriii}{}
\newcommand{\theclassProffessoriv}{}
\newcommand{\classProffessor}[1]{\newcommand{\theclassProffessor}{#1}}
\newcommand{\classProffessorii}[1]{\renewcommand{\theclassProffessorii}{#1}}
\newcommand{\classProffessoriii}[1]{\renewcommand{\theclassProffessoriii}{#1}}
\newcommand{\classProffessoriv}[1]{\renewcommand{\theclassProffessoriv}{#1}}

\newcommand{\thetitlemin}{}
\newcommand{\titlemin}[1]{\renewcommand{\thetitlemin}{#1}}

\usepackage{palatino}
\renewcommand{\textsc}[1]{\fontshape{sc} \fontfamily{\sfdefault} \selectfont #1}

\mathchardef\period=\mathcode`.
\DeclareMathSymbol{.}{\mathord}{letters}{"3B}
\DeclareMathOperator{\erf}{erf}
\DeclareMathOperator{\diff}{d}

\allowdisplaybreaks
\newenvironment{conditions}[1][]
  {#1 \begin{tabular}[t]{>{$}r<{$} @{${}={}$} l @{$\quad$}r @{  }l}}
  {\end{tabular}\\}

  \usetikzlibrary{shapes,shadows}
  \tikzstyle{exercisebox} = [draw=black, rectangle, 
  inner sep=10pt, style=rounded corners, drop shadow={fill=gray,
  opacity=1}]
  \tikzstyle{exercisetitle} =[fill=white]
 \newcommand{\boxexercise}[2]{ 
    \begin{center}
      \begin{tikzpicture}
        \node [exercisebox, fill=black!1!] (box)
        {\begin{minipage}{0.9\textwidth}
            \setlength{\parindent}{1em}
            \footnotesize  \vspace{\belowdisplayskip} #2
          \end{minipage}};
        \node[exercisetitle, right=10pt, draw=black, style=rounded corners] at
        (box.north west) {Exercício: #1};
      \end{tikzpicture}
    \end{center}
  }

  \tikzstyle{examplebox} = [draw=black, rectangle, 
  inner sep=10pt, style=rounded corners, drop shadow={fill=gray,
  opacity=1}]
  \tikzstyle{exampletitle} =[fill=white]
 \newcommand{\boxexample}[2]{ 
    \begin{center}
      \begin{tikzpicture}
        \node [examplebox, fill=green!10!] (box)
        {\begin{minipage}{0.9\textwidth}
            \setlength{\parindent}{1em}
            \footnotesize \vspace{\belowdisplayskip} #2
          \end{minipage}};
        \node[exampletitle, right=10pt, draw=black, style=rounded corners] at
        (box.north west) {Exemplo: #1};
      \end{tikzpicture}
    \end{center}
  }

\onehalfspacing
\newcommand{\code}[1]{\texttt{#1}}

\renewcommand{\lstlistingname}{Código}
\lstset{ language=Matlab,
  basicstyle=\ttfamily,
  basicstyle=\fontfamily{pcr}\fontseries{m}\selectfont\footnotesize,
  breaklines=true,
  columns=fullflexible,
  commentstyle=\color[rgb]{0,0.5,0},
  numbers=left,
  showstringspaces=false,
  morekeywords={matlab2tikz},
  keywordstyle=\color{blue},%
  numberstyle=\tiny\sffamily\color{black},
  frame=tb,
  stringstyle=\color[rgb]{0.5,0,0.5},
  numberstyle=\fontfamily{pcr}\fontseries{m}\selectfont\tiny,
  aboveskip=10pt,belowskip=20pt,
  }

\usepackage{lastpage}
\usepackage{fancyvrb}
% redefine \VerbatimInput
\RecustomVerbatimCommand{\VerbatimInput}{VerbatimInput}%
{fontsize=\scriptsize,frame=single,
rulecolor=\color{lightgray!80},
}

\usepackage{zref-abspage,zref-lastpage}
\makeatletter
  \newcommand*{\iffancylastpage}{%
    \ifnum\zref@extractdefault{LastPage}{abspage}{-1}%
        =\numexpr\value{abspage}+1\relax
      \expandafter\@firstoftwo
    \else
      \expandafter\@secondoftwo
    \fi
  }%
\makeatother

% o novo maketitlepage
\renewcommand{\maketitle}{
% Primeira página de título  
\pagestyle{empty}
\begin{center}   
\textsc \large  
Universidade Federal do Rio Grande do Sul \\
Escola de Engenharia \\
Departamento de Engenharia Química \\
Programa de Pós--Graduação em Engenharia Química
\vfill
\Large \textsc \theclass \\
\begin{varwidth}[t]{\textwidth}
Prof:~\theclassProffessor\\
\ifthenelse{\equal{\theclassProffessorii}{}}{}{Prof:~\theclassProffessorii\\}
\ifthenelse{\equal{\theclassProffessoriii}{}}{}{Prof:~\theclassProffessoriii\\}
\ifthenelse{\equal{\theclassProffessoriv}{}}{}{Prof:~\theclassProffessoriv}
\end{varwidth}
\vfill
\huge \bfseries \textsc  
\thetitle
\vfill 
\begin{varwidth}[t]{\textwidth}
\Large \bfseries \textsc 
\theauthor\\
\ifthenelse{\equal{\theauthorii}{}}{}{\theauthorii\\}
\ifthenelse{\equal{\theauthoriii}{}}{}{\theauthoriii\\}
\ifthenelse{\equal{\theauthoriv}{}}{}{\theauthoriv}
\end{varwidth}
\vfill
\large \textsc Porto Alegre, RS \\ \today
\end{center}
\clearpage
\setcounter{page}{1}
\pagestyle{fancy}
\lhead{\slshape\bfseries\large\MakeUppercase\thetitlemin}
% \lfoot{\iffancylastpage{\color{lightgray}Certified by MacGyver}{}}%
\rfoot{\iffancylastpage{Provided by \Large \LaTeX}{}}%
} % end maketitle 


%\usepackage[style=authortitle-icomp,natbib=true,sortcites=true,block=space]{biblatex}

% float to the end of document (draft for revision)
%\usepackage[tablesfirst]{endfloat}

\usepackage{natbib}
\usepackage{soul}


\begin{document}
\begin{frontmatter}
\title{An alternative reference fluid for the COSMO-SAC model}
\author{Rafael de P. Soares\corref{cor1}}
\ead{rafael.pelegrini@ufrgs.br}

\address{Departamento de Engenharia Qu\'imica, Escola de
Engenharia, Universidade Federal do Rio Grande do Sul,
Rua Engenheiro Luis Englert, s/n, Bairro Farroupilha, CEP 90040-040, Porto
Alegre, RS, Brazil\\July 2013}

\cortext[cor1]{Corresponding author. Tel.:+55 51 33083528; fax: +55 51 33083277}

\end{frontmatter}
%\section*{Introduction}
%\label{sec:intro}

In the COSMO-SAC \citep{Lin:2002} model of \citeauthor{Lin:2002}, the \emph{activity coefficient} of a
segment $\Gamma_m$ is obtained using a reference fluid where all segments
have an identical chemical potential $\mu^0(0)$.
In the present work we show a new derivation for the activity coefficient of a segment
using the reference of an \emph{ideal} fluid with zero interaction energies but
not identical chemical potentials.

We start from the Eq.~(9) of \cite{Lin:2002}, which is also similar to the one originally derived
by \citeauthor{Klamt:1995}\cite{Klamt:1995}:
\begin{equation}\label{eq:mu}
	\frac{\mu_m}{kT} =-\ln\left[\sum_n
	\exp\left(\frac{-E_{m,n} + \mu_n}{kT}\right)\right] + \ln{p_m}
\end{equation}
where $\mu_m$ is the chemical potential of a segument $m$; $E_{m,n} = E_{n,m}$ is the interaction
energy for the contact between the segments $m$ and $n$; $p_m$ is the
probabylity of finding a segment $m$ in the mixture; and $T$ is the temperature.

Now let us consider an \emph{ideal} fluid where all the interations are zero, $E_{m,n}=0$,
resulting in:
\begin{equation}\label{eq:mu0}
	\frac{\mu_m^0}{kT} =-\ln\left[\sum_n
	\exp\left(\frac{\mu_n^0}{kT}\right)\right] + \ln{p_m}
\end{equation}

One could argue that, if there is no interaction, the chemical potential
of a segment $\mu_m^0$ should not depend on the chemical potential
of other segments $\mu_n^0$.
Indeed, since $\sum_n p_n = 1$, one can verify that $\mu_m^0/kT =
\ln{p_m}$ is a solution of Eq.~(\ref{eq:mu0}).

Then, if we subtract $\mu_m^0/kT = \ln{p_m}$ from Eq.~(\ref{eq:mu})
and define the activity coefficient of a segment as
$\ln\Gamma_m \equiv (\mu_m - \mu_m^0)/kT$, we get:
\begin{equation}\label{eq:lnGAMMA}
	\ln\Gamma_m =-\ln\left[\sum_n
	\exp\left(\frac{-E_{m,n} + \mu_n}{kT}\right)\right]
\end{equation}

Finally, if we use $\mu_n/kT = \ln\Gamma_n + \mu_n^0/kT =
\ln\Gamma_n + \ln p_n$, it is possible to show that:
\begin{equation}\label{eq:lnGAMMA}
	\ln\Gamma_m =-\ln\left[\sum_n p_n \Gamma_n
	\exp\left(\frac{-E_{m,n}}{kT}\right)\right]
\end{equation}
which is equivalent to Eq.~(10) of \cite{Lin:2002}.

With this derivation it is more clear that many contributions can be added directly
to the interaction energy $E_{m,n}$, not only the usual electrostatic and
hydrogen bond contributions.
Further, that the activity coefficient of a segment $\ln\Gamma_m$ can be used
to compute the residual Helmholtz energy, with respect to a non-interacting
fluid.

\bibliography{ppgeq}

\end{document}
